\chapter{Rate of Convergence}\label{chapt:rate of convergence}
In the previous chapter, we have shown that for an arbitray finite number of orthogonal proections $P_{i}$ for $i=1,2,\cdots$, $T=P_{r}P_{r-1}\cdots P_{1}$, $T^n$ converges to $P_{M}$ strongly as $n\rightarrow \infty$, but we are not sure about how fast it converges. It is crucial to study the rate of convergence when the orthogonal projections are used in applications, such as the \emph{Schwarz Alternating Method}, which we will discuss in the next chapter.


In this chapter, we will give a detailed discussion about the rate of convergence. We first study the dichotomy results from \emph{Deutsch} and \emph{Hundal's} work\cite{DH15} in 2015, and give a proof for a slightly weaker result of the general Hilbert Space case. The second half of the chapter focuses more on the concept of \emph{Friedrichs Angles} which gives a good description for the particular two closed subspaces case for the method of alternating projections.