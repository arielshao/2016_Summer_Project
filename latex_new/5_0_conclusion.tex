\chapter{Conclusion}\label{chapt:conclusion}
In this paper, our main focus was the \emph{Schwarz Alternating Method} for the two-dimensional Poisson's equation with Neumann boundary conditions. In fact, this iterative approach also applies to other different classes of equations such as Stokes equations and nonlinear variants, and more details can be found in \emph{Lion's} 1988 paper \cite{PL88}. Moreover, we can also extend from two subdomains to any fintie number of subdomains. The convergence result follows from the  \emph{von-Neumann Halperin Theorem}, while the algorithm is very similiar to our two dimensional case.

In the numerical analysis section of this paper, we restricted the application of our algorithm to a specific L-shaped domain for simplicity, but I believe that our  Matlab code can also be adapted for any composite domain which is a union of two subdomains with uniform overlapping.


%Furthermore, von Neumann's alternating projections algorithm can also be applied in many other areas, such as interpolation of stochastic processes \cite{•},\cite{•},\cite{•}, reconstruction of images in medicine and geogphysics\cite{•},\cite{•}, and some other applications mentioned by \emph{Deutsch} \cite{De83}.

