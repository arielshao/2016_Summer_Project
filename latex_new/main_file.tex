\documentclass[12pt, a4paper]{book}
\usepackage{amssymb}
\usepackage{amsmath}
\usepackage{amssymb}
\usepackage{amsthm}
\usepackage{fancyhdr}
\pagestyle{fancy}
\usepackage{calc}
\fancyheadoffset[LE,RO]{\marginparsep+\marginparwidth}
\fancyhfoffset[L]{0.01cm}
\fancyhfoffset[R]{0.01cm}
\renewcommand{\chaptermark}[1]{\markboth{#1}{}}
\renewcommand{\sectionmark}[1]{\markright{\thesection\ #1}}
\fancyhf{}
\fancyhead[LE,RO]{\thepage}
\fancyhead[LO]{\rightmark}
\fancyhead[RE]{\leftmark}
\fancypagestyle{plain}%
\fancyhead{} % get rid of headers
\renewcommand{\headrulewidth}{0pt} % and the line
\usepackage{titlesec} 
\usepackage{bbm}
\usepackage{mathrsfs}
\usepackage{xypic}
\usepackage{enumitem}
\setenumerate{label={\rm (\alph{*})}}
\usepackage[colorlinks=true,linkcolor=blue,urlcolor=blue]{hyperref}
\usepackage{graphicx}
\usepackage{geometry}
\geometry{a4paper,left=35mm,right=35mm, top=40mm, bottom=30mm} 
\titleformat{\chapter}[display]
{\bfseries\Large}
{\filright\MakeUppercase{\chaptertitlename} \Huge\thechapter}
{1ex}
{\titlerule\vspace{1ex}\filleft}
[\vspace{1ex}\titlerule]
\renewcommand{\thesection}{\thechapter.\arabic{section}}
\titleformat{\section}{\bfseries\large}{\thesection.}{0.3em}{}
\titlespacing{\section}{0pt}{12pt}{6pt}
\titleformat{\subsection}[runin]{\bfseries}{\thesubsection.}{0.2em}{}
\titlespacing{\subsection}{0pt}{12pt}{6pt}
\numberwithin{equation}{section}
\usepackage{etoolbox}
\makeatletter
\patchcmd{\@thm}{\thm@headfont{\scshape}}{\thm@headfont{\bfseries}}{}{}
\patchcmd{\@thm}{\thm@notefont{\fontseries\mddefault\upshape}}{}{}{}
\makeatother
\newtheorem{theorem}{Theorem}[chapter]
\newtheorem{lemma}[theorem]{Lemma}
\newtheorem*{lemmazorn}{Zorn's Lemma}
\newtheorem{proposition}[theorem]{Proposition}
\newtheorem{remo}[theorem]{Remark}
\newtheorem{corollary}[theorem]{Corollary}
\newtheorem{convention}[theorem]{Convention}
\newtheorem{counterexample}[theorem]{Counterexample}
\newtheorem{hypothesis}[theorem]{Hypothesis}
\newtheorem{question}[theorem]{Question}
\newtheorem{proposition and definition}[theorem]{Proposition and Definition}
\newtheorem{definition}[theorem]{Definition}
\newtheorem{notation}[theorem]{Notation}
\newtheorem{assumption}[theorem]{Assumption}
\newtheorem{Remark}[theorem]{Remark}
\newtheorem{remark}[theorem]{Remark}
\newtheorem{example}[theorem]{Example}
\newtheorem{conjecture}[theorem]{Conjecture}
\numberwithin{equation}{section}
\renewcommand{\baselinestretch}{1.2}\normalsize
\providecommand{\meantmp}[2]{#1\langle{#2}#1\rangle}
\providecommand{\mean}[1]{\meantmp{}{#1}}
\providecommand{\bigmean}[1]{\meantmp{\big}{#1}}
\providecommand{\Bigmean}[1]{\meantmp{\Big}{#1}}
\providecommand{\biggmean}[1]{\meantmp{\bigg}{#1}}
\providecommand{\Biggmean}[1]{\meantmp{\Bigg}{#1}}
\providecommand{\Ma}{\ensuremath{\mathcal{M}^\alpha}}
\providecommand{\Mas}{\ensuremath{\mathcal{M}^\alpha_\sigma}}
\providecommand{\Na}{\ensuremath{\mathcal{N}^{\alpha}}}
\providecommand{\Nares}{\ensuremath{\Na_{\text{res}}}}
\providecommand{\comment}[1]{\vskip.3cm
\fbox{%
\parbox{0.93\linewidth}{\footnotesize #1}}
\vskip.3cm}
\def\Xint#1{\mathchoice
{\XXint\displaystyle\textstyle{#1}}%
{\XXint\textstyle\scriptstyle{#1}}%
{\XXint\scriptstyle\scriptscriptstyle{#1}}%
{\XXint\scriptscriptstyle\scriptscriptstyle{#1}}%
\!\int}
\def\XXint#1#2#3{{\setbox0=\hbox{$#1{#2#3}{\int}$ }
\vcenter{\hbox{$#2#3$ }}\kern-.6\wd0}}
\def\ddashint{\Xint=}
\def\dashint{\Xint-}
\newcommand{\id}{\operatorname{Id}}
\newcommand{\diag}{\operatorname{diag}}
\newcommand{\diam}{\operatorname{diam}}
\newcommand{\cof}{\operatorname{cof}}
\newcommand{\im}{\operatorname{im}}
\newcommand{\bd}{\operatorname{BD}}
\newcommand{\bv}{\operatorname{BV}}
\newcommand{\BV}{\operatorname{BV}}
\newcommand{\BD}{\operatorname{BD}}
\newcommand{\Leb}{\mathbf{L}}
\newcommand{\sbd}{\mathbf{SBD}}
\newcommand{\bdiv}{\mathbf{BD}_{\text{div}}}
\newcommand{\bviv}{\mathbf{BV}_{div}}
\newcommand{\ld}{\operatorname{LD}}
\newcommand{\ep}{\varepsilon}
\newcommand{\di}{\operatorname{div}}
\newcommand{\Lip}{\operatorname{Lip}}
\newcommand{\essential}{\operatorname{esssup}}
\newcommand{\dif}{\operatorname{d}}
\newcommand{\Radonfin}{M_{b}(\Omega,\mathbb{R}_{\text{sym}}^{d\times d})}
\newcommand{\Radon}{M_{b}(\Omega,\mathbb{R}^{d\times d})}
\newcommand{\DIF}{\mathbf{D}}
\newcommand{\spt}{\operatorname{spt}}
\newcommand{\N}{\mathbb{N}}
\newcommand{\X}{\mathbf{X}}
\newcommand{\tr}{\operatorname{tr}}
\newcommand{\eind}{\operatorname{ell ind}}
\newcommand{\cod}{\operatorname{codim}}
\newcommand{\cok}{\operatorname{Coker}}
\newcommand{\R}{\mathbb{R}}
\newcommand{\x}{\mathbf{x}}
\newcommand{\uu}{\mathbf{u}}
\newcommand{\meas}{\operatorname{meas}}
\newcommand{\locc}{\operatorname{loc}}
\newcommand{\sing}{\operatorname{Sing}}
\newcommand{\reg}{\operatorname{Reg}}
\newcommand{\excess}{\operatorname{Ex}}
\newcommand{\brz}{\operatorname{B}(z,r)}
\newcommand{\brx}{\operatorname{B}(x,r)}
\newcommand{\brxo}{\operatorname{B}(x_{0},r)}
\newcommand{\broz}{\operatorname{B}(Z,\rho)}
\newcommand{\dist}{\operatorname{dist}}
\newcommand{\supp}{\operatorname{supp}}
\newcommand{\gp}{\mathbf{g}_{p}}
\newcommand{\range}{\operatorname{Ran}}
\newcommand{\iu}{\operatorname{i}}
\newcommand{\trace}{\operatorname{Tr}}
\newcommand{\diameter}{\operatorname{diam}}
\newcommand{\capa}{\operatorname{Cap}}
\newcommand{\ranka}{\operatorname{rank}}
\newcommand{\ranga}{\operatorname{Ran}}
\newcommand{\bvc}{\operatorname{BV}_{\operatorname{c}}}
\newcommand{\E}{\mathbf{E}}
\newcommand{\wlim}{\operatorname{w^{*}-}\lim_{\varepsilon\searrow 0}}
\newcommand{\wstar}{\stackrel{*}{\rightharpoonup}}
\newcommand{\ball}{\operatorname{B}}
\renewcommand{\rho}{\varrho}
\newcommand{\Cc}{\operatorname{C}_{\operatorname{c}}}
\renewcommand{\c}{\operatorname{c}}
\renewcommand{\hom}{\operatorname{Hom}}
\newcommand{\W}{\operatorname{W}}
\newcommand{\ellind}{\operatorname{ellind}}
\newcommand{\spano}{\operatorname{span}}
\newcommand{\Con}{\operatorname{C}}
\newcommand{\ad}{\mathbb{A}[D]u}
\newcommand{\gm}{\operatorname{GM}}
\newcommand{\finfty}{f^{\infty}}
\newcommand{\yhat}{\ensuremath{\hat{y}}}
\newcommand{\bspq}{\operatorname{B}_{p,q}^{s}}
\newcommand{\lpw}{\operatorname{L}_{\omega}^{p}}
\newcommand{\lp}{\operatorname{L}^{p}}
\newcommand{\lloc}{\operatorname{L}_{\operatorname{loc}}^{1}}
\newcommand{\mes}{\operatorname{mes}}
\newcommand{\lebe}{\operatorname{L}}
\newcommand{\sobo}{\operatorname{W}}
\newcommand{\besov}{\operatorname{B}}
\newcommand{\pv}{\operatorname{pv}}
\newcommand{\hold}{\operatorname{C}}
\newcommand{\traceop}{\operatorname{Tr}}
\newcommand{\bvloc}{\operatorname{BV}_{\operatorname{loc}}}
\renewcommand{\epsilon}{\varepsilon}
\newcommand{\lpnorm}{\|\cdot\|_{p,\Omega}}
\newcommand{\Span}{\operatorname{span}}
\newcommand{\aff}{\operatorname{aff}}
\newcommand{\con}{\operatorname{conv}}
\newcommand{\B}{\mathbb B}
\newcommand{\rank}{\operatorname{rank}}
\newcommand{\interior}{\operatorname{int}}
\newcommand{\Le}{\mathscr{L}^{n}}
\newcommand{\so}{\operatorname{SO}}
\newcommand{\Div}{\operatorname{div}}
\newcommand{\orth}{\operatorname{O}}
\newcommand{\spa}{\operatorname{span}}
\newcommand{\proj}{\operatorname{proj}}
\newcommand{\conv}{\operatorname{conv}}
\newcommand\myeq{\mathrel{\overset{\makebox[0pt]{\mbox{\normalfont\tiny\sffamily DCT}}}{=}}}
\newcommand\myeqIP{\mathrel{\overset{\makebox[0pt]{\mbox{\normalfont\tiny\sffamily IP}}}{=}}}
\usepackage{tikz-cd}
\begin{document}
\frontmatter
\begin{titlepage}
\begin{center}
\textsc{\LARGE University of Oxford}\\[1.5cm]
\includegraphics[width=0.25\textwidth]{ox}\\[1cm]
\textsc{\Large OxPDE Summer Project 2016}\\[0.5cm]
\newcommand{\HRule}{\rule{\linewidth}{0.5mm}}
\HRule \\[0.4cm]
{ \huge \bfseries Orthogonal Projections in Hilbert Spaces}\\[0.4cm]
\HRule \\[1.5cm]
\textsc{\Large Aili SHAO} \\
\textsc{\Large Magdalen College}
\vfill
\begin{center} {\large
\emph{Under the Supervision of}}\\
\end{center}
\begin{center}{\large 
Dr. David \textsc{Seifert}}
\end{center}
% bottom of page:
%{\large Version of \today}
\end{center}
\end{titlepage}
\newpage
\input{0_1_abstract.tex}
\input{0_2_acknowledgement.tex}
\tableofcontents 
\newpage
\mainmatter
\input{1_0_introduction.tex}
\chapter{Iterated Products of Orthogonal Projections in Hilbert Space}\label{chapt:Iterated_Products_of_Projections_in_Hilbert_Space}In this chapter, we focus on the convergence of the iterated products of orthogonal projections in Hilbert Space. This classical result was first estabilished for two orthogonal projections by \emph{J.Von Neumann}\cite{JN33} in 1933 in the form of the following theorem:
\begin{theorem}
Let $X$ be a Hilbert Space, and $M_1,M_2$ be closed subspaces of $X$. If $P_{M_i}$ is an orthogonal proection on $M_i$ for eacu $i=1,2$, and $P_{M}$ is the orthogonal projection on the closed subspace $M=M_1\cap M_2$, then for each $x\in X$,
$$\lim_{n\rightarrow\infty}(P_{M_1}P_{M_2})^n(x)=P_{M}(x).$$
\end{theorem}

We can simply demonstrate this beautiful result in the two-dimensional case:

Let $M_1$ and $M_2$ be the two straight lines as shown in the diagram below, and $x$ be an arbitrary element from $\mathbb{R}^2$. If we define an alternating sequences $\{x_n\}$ by
$$x_0=x,$$
$$x_{2n+1}=P_{M_1}(x_{2n}),$$
$$x_{2n}=P_{M_2}(x_{2n-1})=(P_{M_1}P_{M_2})^n(x_0)), $$
then $\lim_{n\rightarrow\infty}x_n=P_{M}(x).$
\begin{figure}[h]
\includegraphics[width=140mm,scale=0.8]
{2d_demo.png}
\end{figure}
\par
 The same theorem was also found by \emph{Nakano} \cite{NN53} in 1953 and \emph{Wiener}\cite{NW55} in 1955. Indeed, the result is also true for any arbitrary finite number of orthogonal projections as shown in \emph{Halperin's} work \cite{IH62} in 1962.

\input{2_1_Convergence.tex}
\chapter{Rate of Convergence}\label{chapt:rate of convergence}
In the previous chapter, we have shown that for an arbitray finite number of orthogonal proections $P_{i}$ for $i=1,2,\cdots$, $T=P_{r}P_{r-1}\cdots P_{1}$, $T^n$ converges to $P_{M}$ strongly as $n\rightarrow \infty$, but we are not sure about how fast it converges. It is crucial to study the rate of convergence when the orthogonal projections are used in applications, such as the \emph{Schwarz Alternating Method}, which we will discuss in the next chapter.


In this chapter, we will give a detailed discussion about the rate of convergence. We first study the dichotomy results from \emph{Deutsch} and \emph{Hundal's} work\cite{DH15} in 2015, and give a proof for a slightly weaker result of the general Hilbert Space case. The second half of the chapter focuses more on the concept of \emph{Friedrichs Angles} which gives a good description for the particular two closed subspaces case for the method of alternating projections.
\input{3_1_dichotomy_results.tex}
\section{Friedrichs Angles}\label{sec:friedrichs}
In order to apply the \emph{Von-Neumann Halperin Theorem} in the method of alternating projections, it is important to know the \emph{Friedrichs Angle} which is defined in the following sense:
\begin{definition}
Let $X$ be a Hilbert space, and $M_1$, $M_2$ are two closed subspaces of $X$ with intersection $M:=M_1\cap M_2$. The \emph{Friedrichs angle} between $M_1$ and $M_2$ is defined to be the angle in $[0,2\pi]$ whose cosine is given by
$$ c(M_1,M_2)=\sup\{|\left\langle x,y\right\rangle|\colon x\in M_{1}\cap M^{\perp}, \|x\|\leq 1, y\in M_2\cap M^{\perp},\|y\|\leq 1\}.$$
\end{definition}

Now we can deduce a relationship between the rate of convergence and the Friedrichs angle between the two closed subspaces $M_1$, $M_2$.
\begin{theorem}(\cite{KW88})\label{t1}
Let $X$ be a Hilbert space, and $M_1, M_2$ and $M$ be defined as above. If $P_1$ and $P_2$ are the orthogonal projections onto $M_1$ and $M_2$ respectively, and $P_M$ is the orthogonal projection onto $M$, then for each $n\in\mathbb{N}$, we have
$$\|(P_2P_1)^n-P_{M}\|=c(M_1,M_2)^{2n-1}.$$
\end{theorem}
Before proving the main theorem, we introduce some fundamental results first.
\begin{lemma}\label{l1}
Let $Q_i:=P_i(I-P_{M})$ for each $i=1,2$, then $$(P_2P_1)^n-P_M=(Q_2Q_1)^n.$$
\begin{proof}
\begin{align*}
(P_2P_1)^n-P_M &= (P_2P_1)^n-(P_2P_1)^n P_M\\
{}&=(P_2P_1)^n(I-P_M)\\
{}&=(P_2P_1)^n P_{M^{\perp}}\\
{}&=(P_2P_1)^n P_{M^{\perp}}^n  \mbox{ as } P_{M^{\perp}}^2=P_{M^{\perp}} \\
{}&= (P_2P_1 P_{M^{\perp}})^n   \mbox{ as } P_2P_1 \mbox{ commutes with } P_{M^{\perp}}\\
{}&=(P_2 P_{M^{\perp}} P_1 P_{M_{\perp}})^n\\
{}&=(Q_2Q_1)^n
\end{align*}
where the second last inequality follows from $P_i$ commuting with $P_{M^{\perp}}$.
\end{proof}
\begin{lemma}\label{l2}
If $T\in B(X)$ with $X$ being a Hilbert space is a self-adjoint linear operator, then for each $n\in\mathbb{N}\cup\{0\}$,
$$\|T^n\|=\|T\|^n.$$
\end{lemma}
\begin{proof}
Note that if $T$ is self-adjoint, we have $\|T^2\|=\|T\|^2$. (B4.2 Hilbert space lecture notes) 
Similarly, $\|T^4\|=\|T^2\|^2=\|T\|^4$. By induction, the result is true for $n=2^m$ with $m\in \mathbb{N}\cup\{0\}$.  

For any $n\in\mathbb{N}$ not in this form, we can write $n=2^m-r$ for some $m,r\in\mathbb{N}\cup\{0\}$, then 
$\|T\|^{n+r}=\|T^{n+r}\|\leq \|T^{n}\|\|T^r\|\leq\|T^n\|\|T\|^r$. This gives $\|T\|^n\leq \|T^n\|$, and thus $\|T^n\|=\|T\|^n$.
\end{proof}
\end{lemma} 
\begin{lemma}\label{l3}
$c(M_1,M_2)=\|Q_2Q_1\|=\sqrt{\|Q_1Q_2Q_1\|}$.
\end{lemma}
\begin{proof}
By definition, we have
\begin{align*}
c(M_1,M_2)&=\sup\{|\left\langle x,y\right\rangle|\colon x\in M_{1}\cap M^{\perp}, \|x\|\leq 1, y\in M_2\cap M^{\perp},\|y\|\leq 1\}\\
{}&=\sup\{|\left\langle P_{M_1\cap M^{\perp}}x, P_{M_2\cap M^{\perp}}y\right\rangle|\colon \|x\|\leq 1,\|y\|\leq 1\}\\
{}&=\sup\{|\left\langle P_{M_2\cap M^{\perp}}P_{M_1\cap M^{\perp}}x, y\right\rangle|\colon \|x\|\leq 1,\|y\|\leq 1\}\\
{}&=\|P_{M_2\cap M^{\perp}}P_{M_1\cap M^{\perp}}\|\\
{}&=\|(P_{M_2}P_{M^{\perp}})(P_{M_1}P_{M^{\perp}})\| \mbox{ as } P_{i} \mbox{ commutes with } P_{M^{\perp}} \mbox{ for } i=1,2 \\
{}&=\|Q_2Q_1\|.
\end{align*}
Also, $\|Q_2Q_1\|^2=\|(Q_2Q_1)^{\ast}Q_2Q_1\|=\|Q_1Q_2Q_2Q_1\|=\|Q_1Q_2Q_1\|$, then the second inequality follows.
\end{proof}
Now we are ready to prove Theorem \ref{t1}:
\begin{proof}
By Lemma \ref{l1}, $\|(P_2P_1)^n-P_M\|=\|(Q_2Q_1)^n\|$. Since $((Q_2Q_1)^n)^{\ast}=(Q_1Q_2)^n$, we have 
$$\|(Q_2Q_1)^n\|^2=\|(Q_1Q_2)^n(Q_2Q_1)^n\|=\|(Q_1Q_2Q_1)^{2n-1}\|.$$

As the operator $Q_1Q_2Q_1$ is self-adjoint, it follows from Lemma \ref{l2} that $$\|(Q_1Q_2Q_1)^{2n-1}\|=\|Q_1Q_2Q_1\|^{2n-1}.$$

By applying Lemma \ref{l3}, the result then follows.
\end{proof}
Therefore, $\|(P_2P_1)^n-P_M\|$ converges to $0$ expoentially fast if and only if $c(M_1,M_2)<1$.
\par

\begin{theorem}
Let $M_i$ for $1\leq i\leq r$ be closed subspaces in the Hilbert space $X$, and $M:=\bigcap_{i}^r M_i.$ Let $P_{i}$ and $P_{M}$ be the orthogonal projections onto $M_i$ and $M$ respectively. If $T=P_rP_{r-1}\cdots P_1$, then $\|T^n-P_M\|$ converges to $0$ exponetially fast if and only if $\mathrm{Im}(I-T)$ is closed.
\begin{proof}
Since $M:=\bigcap_{i}^r M_i$ is closed, $X=M\oplus M^{\perp}$. We have proved that $M=\mathrm{Ker}(I-T^{\ast})$, it follows that $M^{\perp}=\overline{\mathrm{Im}(I-T)}$. Let $	Y=\mathrm{Im}(I-T)$, and $Z=\overline{Y}$.

In the proof of the \emph{dichotomy results}, we have showed that the convergence is exponentially fast if and only if $r(S)<1$ where $S:=T\mid_{Z}=TP_{M^{\perp}}$.
Then $I-S\colon Z\to Z$ has trivial kernel since if
$(I-S)x=0$ for some $x\in Z$, we have $x=Sx=Tx$, that is, $x\in M^{\perp}\cap M=\{0\}$. 
We also have $\mathrm{Im}(I-S)=Y$. For each $y\in \mathrm{Im}(I-S)$, there exists $x\in Z$ such that 
\begin{align*}
y&=(I-S)x\\
{}&=x-Sx \\
{}&=x-TP_{M^{\perp}}x \\
{}&=x-Tx  \mbox{ as } x\in Z=M^{\perp}\\
{}&=(I-T)x. 
\end{align*}
This implies that $\mathrm{Im}(I-S)\subseteq Y$.
If $y\in Y=\mathrm{Im}(I-T)$, there exists $x\in X$ such that 
\begin{align*}
y&=P_{M^{\perp}}y \mbox{ as } y\in Y\subseteq M^{\perp}\\
{}&=P_{M^{\perp}}(I-T)x \\
{}&=P_{M^{\perp}}x-P_{M^{\perp}}Tx \\
{}&=P_{M^{\perp}}x-P_{M^{\perp}}^{2}Tx \\
{}&=P_{M^{\perp}}(I-P_{M^{\perp}}T)x\\
{}&=P_{M^{\perp}}(I-TP_{M^{\perp}})x \mbox{ as } T \mbox{ commutes with } P_{M^{\perp}} \\
{}&=P_{M^{\perp}}(I-S)x \\
{}&=(I-S)x.
\end{align*}
This shows that $Y\subseteq \mathrm{Im}(I-S)$.
So $I-S$ is a bounded bijiection from $Z$ onto $Y$. By \emph{Inverse Mapping Theorem}, $I-S$ is invertible if and only if $Y=Z$. So $1\in \sigma(S)$ if and only if $Y\neq Z$. That is $r(S)<1$ if and only if $Y=Z$.
\end{proof}
\end{theorem}
\input{4_0_Schwarz_Alternating_Method.tex}
\section{One Dimensional Case with Friedrichs Angles Calculated}\label{sec:one-d_case} 

Take $\Omega=(0,1)\subset\mathbb{R}$, and decompose $\Omega$ into two subdomains $\Omega_1:=(0,r)$, $\Omega_2:=(s,1)$ with $s\leq r$ such that $\Omega=\Omega_1\cup\Omega_2$.

In this section, we aim to apply the \emph{Schwarz Alternating Method} in solving the one-dimensional Neumann problem of the following form:
$$\begin{cases}
-u''+u=f  \mbox{ in } \Omega, \\
u'(0)=u'(1)=0\\
\end{cases}$$
where $\Omega=\Omega_1\cup\Omega_2$.

Note that for the one-dimensional case, the weak formulation of the solution of the above equation concides with the classical solution. In order to prove this claim, we need the \emph{Fundamental Lemma of Calculus of Variations}:
\begin{lemma}\label{FTCV}
For $f, g\in C(0,1)$ such that 
$$\int_{0}^{1} (f\varphi +g\varphi ') dx=0$$
for all $\varphi\in C_{0}^{\infty}(0,1)$, we have $g\in C^1(0,1)$ and $g'=f$.
\end{lemma}

The weak formuation of the solution is 
$$\int_{0}^1 u'\varphi '+u\varphi dx=\int_{0}^1 f \varphi dx$$ for all $\varphi\in C^{\infty}(0,1)$. Let $v\in C^1(0,1)$ be the primitive function of $u$, that is, $v(x)=\int_{0}^{x}u(s) ds$, while $\phi\in C_{0}^{\infty}(0,1)$ satisfy $\varphi=\int_{0}^{\infty} \phi(s) ds$. Note that 
$$\int_{0}^{1} u'\varphi ' dx=\left[ u \varphi '\right]_{0}^{1}-\int_{0}^{1} u\varphi'' dx=-\int_{0}^{1} u\varphi '' dx$$ for all $\varphi' \in C_{0}^{\infty}$. Since $\varphi'=\phi$, we have 
$$\int_{0}^{1} u'\phi=-\int_{0}^{1} u\phi ' dx$$.
Then 
\begin{align*}
\int_{0}^{1} f\varphi dx&=\int_{0}^{1} u'\varphi '+u \varphi dx\\
{}&=-\int_{0}^{1} u\phi' dx +\left[v \varphi\right]_{0}^{1}-\int_{0}^{1} v\phi dx \\
{}&=-\int_{0}^{1} u\phi' +v\phi dx
\end{align*}
If $F$ is the primitive function of $f$, say $F(x)=\int_{0}^{x} f(s) ds$, then 
$$\int_{0}^{1} f\varphi dx =\left[ F\varphi\right]_{0}^{1}-\int_{0}^{1} F \phi dx.$$
Rearranging gives 
$$\int_{0}^{1} (u\phi'+(v-F)\phi) dx.$$

By Lemma \ref{FTCV}, we have $u\in C^1$ and $u'=v-F$. Since $v, F\in C^1$, $u\in C^2$.
\input{4_1_1_the_schwarz_alternating_method.tex}
\subsection{Calculation of the Friedrichs Angles}\label{sec:friedrichs_1_d_case}\hfill

Note that $Y=Y_1+Y_2$ is closed, and $Y=X$ if $s<r$ while $Y=\mathrm{Ker}\phi_r$ if $r=s$, where $\phi_r\in X^{\ast}$ is defined by $\phi_r(u)=u(r)$ for $u\in X$. By \emph{Riesz Representation Theorem}, there exists unique $v_r$ such that 
$\phi_r(u)=\left\langle u, v_r\right\rangle _{H^1}$, that is

\begin{align*}
u(r)&=\int_{0}^{1} u{v_r}+ u'{v_r}' dx\\
{}&=\int_{0}^{1} u{v_r} dx+\left[u{v_r}'\right]_{0}^{1}-\int_{0}^{1} u {v_r}''dx\\
{}&=\int_{0}^{1} u({v_r}-{v_r}'')dx+\left[u{v_r}'\right]_{r^{+}}^{1}+\left[u{v_r}'\right]_{0}^{r^{-}}.\\
\end{align*}

We require that 
$$\begin{cases}
{v_r}''=v_r, \\
{v_r}'(r^{-})-{v_r}'(r^{+})=1,\\
v_r(r^{-})=v_{r}(r^{+}),\\
{v_r}'(1)={v_r}'(0)=0.\\
\end{cases}$$ 

By solving the above equations, we have 
$${v_r}=\begin{cases}
\frac{\cosh(1-r)}{\sinh(1)}\cosh x, \mbox{ for } 0<x<r,\\
\frac{\cosh(r)}{\sinh(1)}\cosh(1-x), \mbox{ for } r<x<1.\\
\end{cases}$$ 

Then we know that $$\|\phi_r\|^2=\|v_r\|^2=\left\langle v_r,v_r\right\rangle_{H^1}=\frac{\cosh(r)\cosh(1-r)}{\sinh(1)}.$$

Similiar as before, we let $M_i=Y_{i}^{\perp}$ for $i=1,2$. If $s<r$, $$M^{\perp}=(Y_1^{\perp}\cap Y_{2}^{\perp})^{\perp}=Y_1^{\perp\perp}+Y_2^{\perp\perp}=Y=X,$$
so $M={0}$. 
If $s=r$, we have $$M=M^{\perp\perp}=Y^{\perp}=(\mathrm{Ker}\phi_r)^{\perp}=\left\langle v_r\right\rangle$$.

We now calculate the \emph{Friedrichs number} $c(M_1,M_2)$.

Note that for $u\in H^1(0,1)$, $u-P_1 u$ is orthogonal to $M_1$, that is, $u-P_1u\in Y_1$, then for $x\in (r,1)$, $P_1u(x)=u(x)$. By previous calculation, we know that $M_1=\mathrm{span}\{\cosh x\}$, then $P_1 u(x)=A \cosh x$ for $x\in (0,r)$. By continuity at $r$, $A=\frac{u(r)}{\cosh r}$. Thus
$$P_1u(x)=\begin{cases} u(r)\frac{\cosh(x)}{\cosh(r)}, \mbox{ for } 0<x<r,\\
u(x), \mbox{ for } r<x<1.\\
\end{cases}$$

By a similar argument, we have
$$P_2u(x)=\begin{cases}u(x), \mbox{ for } 0<x<s,\\
u(s)\frac{\cosh(1-x)}{\cosh(1-s)}, \mbox{ for } s<x<1,\\
\end{cases}$$
for all $u\in X$.  For $T=P_2P_1$, we have 

$$Tu(x)=\begin{cases} u(r)\frac{\cosh x}{\cosh r}, \mbox{ for } 0<x<s,\\
u(r)\frac{\cosh(s)\cosh(1-x)}{\cosh r\cosh (1-s)}, \mbox{ for } s<x<1.\\
\end{cases}$$

Thus 
$$Tu=\frac{\sinh(1)\phi_r(u)}{\cosh r\cosh(1-s)}v_s$$
for all $u\in X$.
If $s<r$, $M={0}$, then $P_{M}=0$. 
$$c(M_1,M_2)=\|T-P_M\|=\|T\|=\frac{\sinh(1)\|\phi_r\|\|\phi_s\|}{\cosh r\cosh(1-s)}=\sqrt{\frac{\cosh(1-r)\cosh(s)}{\cosh(1-s)\cosh(r)}}.$$
If $s=r$, then $M=\left\langle v_r\right\rangle$. Since $Tv_r=v_r, T=P_{M}$, and $c(M_1,M_2)=\|T-P_{M}\|=0$.
\input{4_2_N-dimensional_case.tex}
\section{Demonstration of the rate of convergence in Matlab}\label{sec:matlab}
In this section, we analyse the rate of convergence by comparing the plots of error norms against the number of iterations.

In section \ref{sec:dichotomy results}, we have proved the \emph{Von-Neumann Halperin Dichotomy} which states that:

If $M_i$  ($1\leq i\leq r$ ) are closed subspaces in the Hilbert space $X$, and $M:=\bigcap_{i}^r M_i$, then for $T=P_{r}P_{r-1}\cdots P_{1}$ where $P_{i}$ is the orthogonal projection on $M_i$, exactly one of the two following statements holds:
\begin{enumerate}
\item $\sum_{i=1}^r M_{i}^{\perp}$ is closed, then there exists $\alpha\in [0,1)$ and $c\in \mathbb{R}$ such that $\|T^n-P_{M}\|\leq c\alpha^n$ for each $n$.
\item $\sum_{i=1}^r M_{i}^{\perp}$ is not closed, then for each $(r_n)\in c_0$, $r_n\in\mathbb{R}^{+}$, for all $x\in X$, $\|T^n x-P_{M}x\|\neq O(r_n)$.
\end{enumerate}

In fact, we can replace (b) by arbitrarily slow convergence, that is,
for each $(r_n)\in c_0$, $r_n\in\mathbb{R}^{+}$, there exists $x\in X$ such that $\|T^n x-P_{M}x\|\geq r_n$ for each $n\in\mathbb{N}$\cite{DH10a}. 

In \emph{Deusch} and \emph{Hundal's} recent work\cite{DH15}, they suggest  that the $x\in X$ satisfies $\|T^n x-P_{M}x\|\geq r_n$ for all $n$ must be chosen from $X\setminus(M\oplus(M_1^{\perp}+M_2^{\perp}))$.

In the application of Schwarz Altenrating Method for the Neumann problem in the two-dimensional domain, we have showed that $M=\{0\}$, so the $x$ have to be chosen from $H^1(\Omega)\setminus (M_1^{\perp}+M_2^{\perp})$.

In order to test the above conjecture, we claim that for $u\in C(\overline{\Omega})\cap(M_1^{\perp}+M_2^{\perp})$, we have $u(x)=0$ for all problematic points on $\partial\Omega$.  The claim is indeed true due to the following fact: if $u\in C(\overline{\Omega})\cap H^1(\Omega)$, then there exists $\phi_n\in C^{\infty}(\overline{\Omega})$ such that $\|u-\phi_n\|_\infty\rightarrow 0$ as $n\rightarrow \infty$.

Since $u\in C(\overline{\Omega})\cap(M_1^{\perp}+M_2^{\perp})=C(\overline{\Omega})\cap (Y_1+Y_2)=C(\overline{\Omega})\cap Y=C(\overline{\Omega})\cap \overline{Z}$ where $$Z=\{u\in C^{\infty}, u=0 \mbox{ near the problematic points}\},$$ there exists $u_n\in C^{\infty}(\overline{\Omega})\cap Z$ such that $\|u-u_n\|_{\infty}\rightarrow 0$ as $n\rightarrow \infty$.  $u_n=0$ on problematic points implies that $u(x)=\lim_{n\rightarrow} u_n(x)=0$ for all problematic points on $\partial \Omega$. 

Thus for $x_0\in H^1(\Omega)\setminus (M_1^{\perp}+M_2^{\perp})$, we need $x_0\neq 0$ at problematic points, that is, $u_0\neq u$ at problematic points for $u, u_0\in H^1(\Omega)$.


Consider the neumann problem 
$$\begin{cases}
-\Delta u+u=1, \\
\frac{\partial u}{\partial r}=0\\
\end{cases}$$
on a L-shape domain as shown in Figure 4.1.
\begin{figure}[t]
\includegraphics[width=\textwidth]{lshape.jpg} 
\caption{lshape domain}
\end{figure}

It is easy to see that the true solution is $u\equiv 1$. 

If we set $u_0=|\log(\frac{1}{\sqrt{(x^2+y^2)}})|^\alpha$ for $\alpha\in (0,0.5)$, $u_0=\infty$ at the problematic point $(0,0)$, then the surface plot of the solution after $80$ iterations is shown in Figure 4.2. (Details of the Matlab code can be found in Appendix \ref{sec:appA})

\begin{figure}[h]
\includegraphics[width=\textwidth]{verybadinitial_u.png}
\caption{surface plot of the solution $u_n$}
\end{figure}

We plot both $\log(\|u_n-u\|_\infty)$ and $\log(\|u_n-u\|_{H1})$ against the number of iterations with respect to different mesh sizes to demontrate the rate of convergence.

\begin{figure}[h]
\includegraphics[width=\textwidth]{verybadinitial_infnorm.png}
\caption{The plots of $\log(\|u_n-u\|_\infty)$ against number of iterations}
\end{figure}
\begin{figure}[h]
\includegraphics[width=\textwidth]{verybadintial_H1norm.png}
\caption{The plots of $\log(\|u_n-u\|_{H1})$ against number of iterations }
\end{figure}

As we can observe from Figure 4.3 and Figure 4.4, the rate of convergence decays as the number of iterations increase with respect to both the infinity and $H_1$ norms. The results shown on the two error plots are compatible with \emph{Deutsch} and \emph{Hundal's} conjecture.

From the plots we can also deduce that the rate of convergence decreses when we refine the mesh near the problematic point. After we refine the mesh to a certain scale (i.e.Nmesh$=75$ in our plot), the plot will no longer depend on the mesh size.

\begin{figure}[h]
\includegraphics[width=\textwidth]{goodinitial_inf.jpg}
\caption{The plots of $\log(\|u_n-u\|_{\infty})$ against number of iterations }
\end{figure}

Now we start with a good initial guess, $u_0=1-\sin(22x)sin(10y)$, which gives $u_0=1$ on the problematic point. Again, we plot the two error norms against the number of iterations.
\begin{figure}[h]
\includegraphics[width=\textwidth]{goodinitial_H1.jpg}
\caption{The plots of $\log(\|u_n-u\|_{H1})$ against number of iterations }
\end{figure}
 
From \emph{Deutsch} and \emph{Hundal's} conjecture, we expect exponentially fast convergence, that is, straight lines for the plots of both $\log(\|u-u_n\|_{\infty})$ and $\log(\|u-u_n\|_{H1})$ against the number of iterations.
However, the plots shown on Figure 4.5 and Figure 4.6 are bizarre for the first few iterations. I suspect that this is because the numerical solutions we obtained from solving PDEs on different domains are approximate values only. Since the soluitons are pointwise values, we try to decrease our grid size, that is, to increase the number of plots ( change Nplot$=101$ to Nplot=$2001$) to increase their accuracy.  We plot the first few iterations in Figure 4.7 and Figure 4.8 to see whether there is any improvement. 
\begin{figure}[h]
\includegraphics[width=\textwidth]{smallgridsize.png}
\caption{The plots of $\log(\|u_n-u\|_{H1})$ against number of iterations }
\end{figure}
\begin{figure}[h]
\includegraphics[width=\textwidth]{smallgridsize2.png}
\caption{The plots of $\log(\|u_n-u\|_{\infty})$ against number of iterations }
\end{figure}
\par
Surprisingly, the plots are inconsistent with what we expect as well. In fact, the plots are reasonable in some sense since the errors are quickly within machine precision. 

\chapter{Conclusion}\label{chapt:conclusion}
In this paper, our main focus was the \emph{Schwarz Alternating Method} for the two-dimensional Poisson's equation with Neumann boundary conditions. In fact, this iterative approach also applies to other different classes of equations such as Stokes equations and nonlinear variants, and more details can be found in \emph{Lion's} 1988 paper \cite{PL88}. Moreover, we can also extend from two subdomains to any fintie number of subdomains. The convergence result follows from the  \emph{von-Neumann Halperin Theorem}, while the algorithm is very similiar to our two dimensional case.

In the numerical analysis section of this paper, we restricted the application of our algorithm to a specific L-shaped domain for simplicity, but I believe that our  Matlab code can also be adapted for any composite domain which is a union of two subdomains with uniform overlapping.


%Furthermore, von Neumann's alternating projections algorithm can also be applied in many other areas, such as interpolation of stochastic processes \cite{•},\cite{•},\cite{•}, reconstruction of images in medicine and geogphysics\cite{•},\cite{•}, and some other applications mentioned by \emph{Deutsch} \cite{De83}.


\chapter{Appendices}\label{chapt:appendices}
\input{6_1_appA.tex}
\input{7_0_bibiliography.tex}
\end{document}