\section{The Main Theorem}\label{chapt:Theorem1_Aili}
\begin{lemma}\label{1}
Let $\mathcal{H}$ be a Hilbert space and $P_j$ be the orthogonal projection onto the closed subspace $M_j$ of the Hilbert space $\mathcal{H}$ for each $1\leq j\leq k$. If $T=P_k\cdots P_1$, then $\|T^n x-T^{n+1}x\|\rightarrow 0$ as $n\rightarrow\infty$.
\begin{proof}
For any $x\in\mathcal{H}$, we have $$\|T^{n+1}x\|\leq \|T\|\|T^{n}x\|\leq \|P_k\|\cdots \|P_1\|\|T^n x\|\leq \|T^n x\|,$$ This shows that $\{\|T^n x\|\}$ is a decreasing sequence in $\mathbb{R}$ that is bounded below, thus it converges. In particular, $$\|T^n x\|^2-\|T^{n+1} x\|^2\rightarrow 0$$ as $n\rightarrow \infty$. Recall the Pythagorean Theorem which gives for any orthogonal projection $P$ and any $x\in\mathcal{H}$,$$\|x-Px\|^2=\|x\|^2-\|Px\|^2.$$
Define $Q_0=I$ and for $j=1,2,\cdots k, Q_j=P_jQ_{j-1}$, so that $Q_k=T$, then
\begin{align*}
\|T^n x-T^{n+1}x\|^2&=\|\sum_{j=0}^{k-1}(Q_jT^n x-Q_{j+1}T^n x\|^2\\
&\leq \{\sum_{j=0}^{k-1}\|(Q_jT^n x-Q_{j+1}T^n x\|\}^2 \mbox{(By triangle inequality)}\\
&\leq \left(\sum_{j=0}^{k-1}1 \right) \left( \sum_{j=0}^{k-1} \|(Q_j T^n x-Q_{j+1}T^n x\|^2 \right) \mbox {(By Cauchy-Schwarz inequality)}\\
&=k\left( \sum_{j=0}^{k-1} \|(Q_j T^n x-Q_{j+1}T^n x\|^2 \right)\\
&=k\left(\sum_{j=0}^{k-1} \|(Q_j T^n x\|^2-\|Q_{j+1}T^n x\|^2 \right)\mbox{ By Pythagorean Theorem}\\
&=k \left(\|Q_0 T^n x\|^2-\|Q_k T^n x\|^2 \right)\\
&=k \left(\|T^n x\|^2-\|T^{n+1} x\|^2 \right)\\
&\rightarrow 0\\ 
\end{align*}
 as $\|T^n x\|^2-\|T^{n+1} x\|^2\rightarrow 0.$
\end{proof}
\end{lemma}
\begin{theorem}
If $P_{M}$ be the orthogonal projection onto the intersection $M=M_1\cap M_2\cdots \cap M_k$, then $\|T^n x-P_{M}\|\rightarrow 0$ as $n\rightarrow\infty.$
\begin{proof}
First note that
\begin{align*}
\mathcal{H}&=\left(Im(I-T)\right)^{\perp}\oplus\left(Im(I-T)\right)^{\perp\perp} \mbox{ (Projection Theorem)}\\
&=\left(Im(I-T)\right)^{\perp}\oplus\overline{Im(I-T)^{\perp}}\\
&=Ker(I-T^{\ast})\oplus\overline{Im(I-T)^{\perp}}\\
\end{align*}

By Lemma \ref{1} $\|T^n(I-T)x\|=\|T^n x-T^{n+1}x\|\rightarrow 0$ as $n\rightarrow\infty$, then $\|T^n y\|\rightarrow 0$ for $y\in Im(I-T)$. As $\|T^n\|\leq\|T\|^n\leq (\|P_k\|\cdots \|P_1\|)^n\leq 1$, $T^n$ is a bounded linear operator, then $\|T^n y\|\rightarrow 0$ for $y\in \overline{Im(I-T)}$.
If we can show that
\begin{equation}\label{2}
Ker(I-T^{\ast})=M_1\cap M_2\cdots \cap M_k,
\end{equation} 
then for each $x\in \mathcal{H}$, we have $x=y+z$ such that $y\in Im(I-T)$ while $z\in M$, and then
$$\|T^n x-z\|=\|T^n(y+z)-z\|\leq \|T^n y\|+\|T^n z-z\|\rightarrow 0 $$
as $n\rightarrow \infty$.
Take $z=P_M(x)$, then the theorem follows.

\par
To show (\ref{2}), it suffices to show that $T^{\ast}x=x$ if and only if $P_i x=x$ for each $1\leq i\leq k$. If $P_i x=x$ for each $1\leq i\leq k$, then $T^{\ast}x=(P_k\cdots P_1)^{\ast}=P_1^{\ast}\cdots P_k^{\ast}x=P_1P_2\cdots P_k x=x$. Conversely, if $P_ix\neq x$ for some $1\leq i\leq k$, then $\|T^{\ast}x\|=\|P_1P_2\cdots P_k x\|\leq \|P_i x\|< \|x\|$, that is, $T^{\ast}x\neq x$.
\end{proof}

\end{theorem}
\end{document}
 
